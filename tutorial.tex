\documentclass[11pt,a4paper,final]{article}
\usepackage[utf8]{inputenc}
\usepackage[english]{babel}
\usepackage{amsmath}
\usepackage{amsfonts}
\usepackage{amssymb}
\usepackage{graphicx}
\usepackage{hyperref}
\usepackage[left=2cm,right=2cm,top=2cm,bottom=2cm]{geometry}
\begin{document}
\title{Reveal.js Tutorial}
\author{by Kostas and Craig}
\maketitle

In this tutorial we will create a presentation by constructing everything from scratch: HTML, CSS, Markdown and JavaScript.
Before we start the tutorial it is useful to have a quickly go through the main elements of web-page building.

\section{Crash-course}
These \textit{cheat-sheets} have been built from resources available on \emph{CodeAcademy} and the \emph{Mozilla Developer Network} which are also recommended for more in-depth training and resources. An excellent online book is also available on \url{http://eloquentjavascript.net/}.

We decided that the quickest way to approach this is to provide commented samples so we can quickly build a simple web-page.

\subsection{HTML}
HTML (\emph{HyperText Markup Language} is used by browsers to communicate with each other and display data in an attractive fashion. It has a tree-like structure similar to that of XML. Each branch defines a section of a web page is started by some text in brackets \texttt{<branch>} and ends with a \textbf{backward} slash and the text \verb|<\branch>|. Some objects are added in self-stopping branches by using a \textbf{forward} slash before closing the bracket \texttt{<branch/>}

\begin{verbatim}
	<!--Let the browser know this is an HTML doc-->
<!doctype html> 
	<!--<html> does the same as the above but for 
	versions of HTML < 4 (not currently required)-->
<html> 
	<!--The <head> branch specifies information that is 
	first read by the browser to correctly display the page.
	This is where JavaScript source files and CSS files
	are conventionally specified. It also contains other branches
	such as tags/titles/authors/etc used by search engines
	(not currently required)-->
<head>

	<title>Sprout</title>
  <link rel="stylesheet" type="text/css" href="coolStyle.css"/>

</head>
	<!--This is the part where the data goes that is displayed
	to the user.-->
<body>
	<h1>Amazing Presentations Ltd.</h1>
	<div class="wow">
		<h2>Amazing discovery.</h2>
		<p>Something that changes the world.</p>
		<a href="otherPageLocation.html">Lookie here.</a>
	</div>
	<p id = "footer">&copy; Mystwood Publishers Limited</p>
</body>
</html>

\end{verbatim}

\subsection{CSS}
CSS or Cascading Style Sheets are commonly used in web applications to standardise formatting and styles. It is also used in reveal.js to set-up a presentation template.
\begin{verbatim}
body {
	height: 100%;
	margin: 0;
	text-align: center;
	width: 100%;
}

p {
	font-size: 2rem;
}

.wow {
	padding: 250px 0;
	margin: 30px;
  font-family: 'Trebuchet MS', Helvetica, sans-serif;
  background-image: url("cool.jpg");
  background-size: cover;
  color: #ffffff
}

.hero a {
	color: #00FFAA;
	font-size: 24px;
	text-decoration: none;
  font-size: 1.25em;
}

#footer{
  font-size:0.75rem;
}
\end{verbatim}

\subsection{JavaScript}

\subsection{Markdown}
Markdown is very useful for writing content for web-pages books and other types of documents. The language is similar to \LaTeX in the sense that it uses plain text to write the source for a nicely formatted document, but it has the advantage of having very human-readable source code. The best way to write markdown is to use a specialised editor such as Atom, but in this tutorial we will use Brackets with a plugin for MD live generation. Markdown is so simple that an actual tutorial is not required for the basics and the cheat-sheet below should prove sufficient (for a step-by-step 10 minute tutorial check \url{http://commonmark.org/help/tutorial/} ):
\begin{figure}[!HT]
\centering
\includegraphics[scale=0.6]{MD-Tutorial.png}
\end{figure}

\section{Tutorial}

\end{document}
